%!TeX program=LuaLaTeX
%!TeX root=sablona.tex

% Zakladni informace jako kdo je autorem, jaky je nazev prace apod.

\def\autor{%
Petr Novák
}

\def\vedouci{%
Mgr. Lucie Zákravská
}

\def\nazev{%
Vliv počasí na chování ptáků
}

% Typ prace se v dokumentu vytiskne jako podnadpis k nazvu a je implicitne
% velkymi pismeny.

\def\typ{%
seminární práce
}

\def\misto{%
Česká Kamenice
}

\def\organizace{%
Základní škola T. G. Masaryka a gymnázium Česká Kamenice
}

% Abstrakt a klicova slova zpravidla nejsou potreba pro seminarni prace. Pokud
% po vas neni pozadovano, muzete bloky korespondujici k abstraktu a klicovym
% slovum uplne smazat. Sablona se tomuto prizpusobi a nebude je ve finalnim
% dokumentu vykreslovat.

\def\abstrakt{%
Stručné a srozumitelné shrnutí obsahu práce. Abstrakt je nejvýše v rozsahu
jedné stránky, nesmí obsahovat zkratky, kromě zkratek obecně přijatých. V
případě závěrečné práce napsané v jazyce anglickém se český abstrakt neuvádí.
}

\def\klicovaslova{%
seznam klíčových slov, která nejlépe vystihují obsah práce
}

% Nektere organizace vyzaduji presne zneni prohlaseni. V tom pripade ho zde
% muzete zmenit. Pokud po vas neni vyzadovano, lze korespondujici blok smazat.

\def\prohlaseni{%
Prohlašuji, že jsem tuto diplomovou/bakalářskou práci vypracoval(a) samostatně
a použil(a) jen pramenů, které cituji a uvádím v přiloženém seznamu literatury.
}
